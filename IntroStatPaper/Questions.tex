\documentclass[11pt]{amsart}
\usepackage{geometry} 
\geometry{letterpaper}                  

\usepackage{graphicx}
\usepackage{amssymb}
\usepackage{epstopdf}
\DeclareGraphicsRule{.tif}{png}{.png}{`convert #1 `dirname #1`/`basename #1 .tif`.png}

\title{Questions for shiny app}
\author{Chelsey Legacy}

\begin{document}
\maketitle
\section{Sampling Distribution for a Proportion}
1.  Take 1 sample of size 100 with a population proportion of 0.3. Draw a graph of the sample distribution, and write the proportion of those chosen that fell into the ``yes" category. 


Note that the sampling distribution has one single bar at the value for the ``yes" proportion.

2.  Now take 100 samples of size 100 with a population proportion 0.3.  What do you notice about the shape of the sampling distribution now?

3. Now take 1000 samples of size 100 with a population proportion of 0.3.  How would you describe this sampling distribution? What are the mean of this distribution?


\section{Confidence Interval}

1. Draw samples at confidence level 95$\%$, increasing the sample size from 25 to 500 slowly. What do you notice about the length of the intervals as the sample size increased? Why?



2. Now increase the confidence level from $80\%$ to $99\%$ slowly, keeping the sample size at 250. What do you notice about the length of the intervals as the confidence level increased? Why?




\section{Inference for one mean}

1.  Take 1 sample of size 25 with a population mean of 20 and a standard deviation of 3. Draw a graph of the sample distribution, and write the mean of your sample.


Note that the sampling distribution has one single bar at the value of the mean for your sample.

2.  Take 100 samples of size 25 with a population mean of 20 and a standard deviation of 3.  What do you notice about the sampling distribution now?

3. Take 1000 samples of size 25 with a population mean of 20 and a standard deviation of 3.  How would you describe this sampling distribution? What is the mean for this sampling distribution?


\section{Inference for two Proportions}

1. Set the proportion for group 1 to 0.3 and the proportion for group 2 to 0.4. Also, set the number of samples to 1, and draw a sample. What $\hat{p}$ for each sample? 

2. What is your sample statistic for the difference in proportions from question 1?

Note: The sampling distribution has one sample at your sample difference in proportions value.

3. Set number of samples to 100 and redraw. Describe the shape, center, and symmetry of your sampling distribution. 


\section{Inference for two Means}
1. Draw 1 sample with sample size 50 for each group. Set group 1 mean to be 20 and group 2 mean to be 21.  Let the standard deviations for both be 1. What are your sample means for each group?

2. What is your sample statistic for the difference in means from question 1?


Note: The sampling distribution has one sample at your sample difference in means value.

3. Set number of samples to 100 and redraw. Describe the shape, center, and symmetry of your sampling distribution. 




\section{Correlation}
1. Set the correlation to -0.9. Describe the form, strength, and direction of the points.

2. Set the correlation to -0.5. Describe the form, strength, and direction of the points.

3. Set the correlation to 0.5. Describe the form, strength, and direction of the points.

4. Set the correlation to 0.9. Describe the form, strength, and direction of the points.

5. What do you notice about the form and strength as you increased the correlation from -0.9 to 0.9?

\section{Outliers}

1. Select "Dataset 1", and observe the point in red. Click "Fit Line" to get a least squares line to the data. Write the equation of the line below.

2. Next, remove the red point and refit the least squares line by checking the "Fit Line Without Point" box.  Write the equation for that line below. 

3. Do you think that fitting the data without the red point improved the fit of the line to the data? Why?

4. Select "Dataset 2", and observe the point in red. Click "Fit Line" to get a least squares line to the data. Write the equation of the line below.

5. Next, remove the red point and refit the least squares line by checking the "Fit Line Without Point" box.  Write the equation for that line below. 

6. Do you think that fitting the data without the red point improved the fit of the line to the data? Why?

7. Choose "Dataset 3" from the dropdown. Fit the least squares regression line both with and without the red outlier. Do you think the point has high leverage? Do you think the point has a large residual? Explain.

High leverage:

Large residual:

8. Choose "Dataset 4" from the dropdown. Fit the least squares regression line both with and without the red outlier. Do you think the point has high leverage? Do you think the point has a large residual? Explain.

High leverage:

Large residual:


\section{Equation}
1.  Set the intercept to -10 and the slope to 5. Click "Fit Line" and write the equation for the least squares regression line.

2. Write the y-intercept interpretation. Then, check the "Intercept" box and see if your interpretation is correct. Note, the red dot on the plot inidcation the location of the y-intercept.

3. Write the slope interpretation. Then, check the "Slope" box and see if your interpretation is correct.

4.   Set the intercept to 20 and the slope to -15. Click "Fit Line" and write the equation for the least squares regression line.

5. Write the y-intercept interpretation. Then, check the "Intercept" box and see if your interpretation is correct.

6. Write the slope interpretation. Then, check the "Slope" box and see if your interpretation is correct. What is the difference between this interpretation and the one in number 3? Explain this difference.



\section{ANOVA}
Analysis of variance, or ANOVA, is used to compare means among groups.  An ANOVA table can provide a test to see if there is a significant difference among means for multiple groups. If the p-value for the test is significant based on a set alpha level, we know that at least one group mean is significantly different than the other group means.

1. Set all means to 30 and all standard deviations to 1.  State the p-value and conclusion for a test of difference in means.

2.  Set the means to 20, 30, and 30 respectively.  Set all standard deviations to 5.  Draw a sample and state the p-value and conclusion.  

3. Set the mean for groups 1 and 2 to 30, and mean of group 3 to 32.  Set all standard deviations to 5.  Take a sample with sample size 100 and note what the p-value is for the ANOVA test.  Write a conclusion based on your p-value.

4. Next, keep the settings the same as above, but draw a sample of size 10.  Note the p-value, and write a conclusion based on this p-value.  

5.  What generalization can be made regarding sample size and test outcome  after observing what happened in the previous questions?

\newpage



\section{Pre-Questions for Students in 101}

1. Explain what it means to have $90\%$ confidence in a confidence interval.


2. As we increase the confidence level of a confidence interval what happens to the width of the interval?



\section{Post-Questions for Students in 101}

1. Explain what it means to have $90\%$ confidence in a confidence interval.


2. As we increase the confidence level of a confidence interval what happens to the width of the interval?


3. Comments on the Shiny app for confidence level:


4. What is the purpose of an ANOVA test?


5. Comments on the Shiny app for ANOVA tests:




\section{Pre-Questions for Students in 400 Level Stats}

1. What is the y-intercept of a line in words?

2. What is the slope of a line in words?


\section{Post-Questions for Students in 400 Level Stats}

1.  What is the y-intercept interpretation for a line with an intercept of 2?

2. What is the interpretation of the slope for a line with a slope of -3?

3. Comments of the Shiny app section for linear regression.


4. Explain the difference between a sample distribution and a sampling distribution.


5. Comments on the Shiny app for a sampling distribution for means.






\end{document}  
