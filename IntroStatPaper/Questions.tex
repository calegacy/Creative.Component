\documentclass[11pt]{amsart}
\usepackage{geometry} 
\geometry{letterpaper}                  

\usepackage{graphicx}
\usepackage{amssymb}
\usepackage{epstopdf}
\DeclareGraphicsRule{.tif}{png}{.png}{`convert #1 `dirname #1`/`basename #1 .tif`.png}

\title{Questions for shiny app}
\author{Chelsey Legacy}

\begin{document}
\maketitle
\section{Sampling Distribution for a Proportion}
1.  Take 1 sample of size 100 with a population proportion of 0.3. Draw a graph of the sample distribution, and write the proportion of those chosen that fell into the ``yes" category. 

2. What is the mean of the sampling distribution?

3. What do you notice about the mean of the sampling distribution and the proportion in the ``yes" category? Explain this relationship.

4. Draw "Many" samples of size 10 and note the mean and standard deviation of the sampling distribution.

5. Draw "Many" samples of size 1000 and note the mean and standard deviation of the sampling distribution.

6. Explain which sample size gave a better result and explain your choice.



\section{Confidence Interval}

1. Draw samples at confidence level 95$\%$, increasing the sample size from 25 to 500 slowly. What do you notice about the length of the intervals as the sample size increased? Why?



2. Now increase the confidence level from $80\%$ to $99\%$ slowly, keeping the sample size at 250. What do you notice about the length of the intervals as the confidence level increased? Why?




\section{Inference for one mean}

1.  Take 1 sample of size 25 with a population mean of 20 and a standard deviation of 2. Draw a graph of the sample distribution, and write the mean of your sample.

2. What is the mean of your sampling distribution?

3.  What do you notice about the mean of the sampling distribution and the mean of your sample distribution? Explain this relationship. 

4. Draw "Many" samples with with mean 20,standard deviation 2, and sample size 10. What are the mean and standard deviation for your samping distribution?

5. Draw "Many" samples with with mean 20,standard deviation 2, and sample size 1000. What are the mean and standard deviation for your samping distribution?

6. How are the mean and standard deviation different with a sample size of 100 and a sample size of 1000? Explain why this difference occurs. 


\section{Inference for two Proportions}

1. Set the proportion for group 1 to 0.3 and the proportion for group 2 to 0.4. Leave both groups at a sample size of 100. Also, set the number of samples to 1, and click "draw". What $\hat{p}$ for each sample? 

2. What is your sample statistic for the difference in proportions from the sample drawn in part 1?


Note: The sampling distribution has one bar at your sample difference in proportions value.

3. Set number of samples to 100 and redraw. Describe the shape, center, and symmetry of your sampling distribution. 


\section{Inference for two Means}
1. Draw 1 sample with sample size 50 for each group. Set group 1 mean to be 20 and group 2 mean to be 21.  Let the standard deviations for both be 1. What are your sample means for each group?

2. What is your sample statistic for the difference in means from question 1?


Note: The sampling distribution has one sample at your sample difference in means value.

3. Set number of samples to 100 and redraw. Describe the shape, center, and symmetry of your sampling distribution. 




\section{Correlation}
1. Set the correlation to -0.9. Describe the form, strength, and direction of the points.

2. Set the correlation to -0.5. Describe the form, strength, and direction of the points.

3. Set the correlation to 0.5. Describe the form, strength, and direction of the points.

4. Set the correlation to 0.9. Describe the form, strength, and direction of the points.

5. What do you notice about the form and strength as you increased the correlation from -0.9 to 0.9?

\section{Outliers}

1. Plot data and choose to fit a line to that data. Write the equation for that line below and provide the $R^2$ value. 

2. Add an outlier at the point (2,2). The fit the line to the data. What is your new linear equation and $R^2$ value?

3. Which equation would you choose to fit the data with the outlier? Explain.


\section{Equation}

1. Using the default settings calculate the linear equation for the data shown. Show your work. You can check your answer by selecting "Fit Line" when you are done. 


\section{ANOVA: Means}
Analysis of variance, or ANOVA, is used to compare means among groups.  An ANOVA table can provide a test to see if there is a significant difference among means for multiple groups. If the p-value for the test is significant based on a set alpha level, we know that at least one group mean is significantly different than the other group means.

1. Set all means to 30 and click "draw".  State the p-value and conclusion for a test of difference in means.

2.  Set the means to 20, 30, and 30 respectively.  Draw a sample and state the p-value and conclusion.

3. Explain breifly why there was a difference in the results for questions 1 and 2. 




\section{ANOVA: Standard Deviation}

1.  Set all standard deviations to 10.  What is the p-value for the ANOVA test?  Write a conclusion based on your p-value.

2. Set one standard deviation to 5, another to 25, and the last to 50.  What is the p-value for the ANOVA test?  Write a conclusion based on your p-value.

3. Does it appear that there is a relationship between standard deviation and the result of the F test with all means and sample sizes the same?

\section{ANOVA: Sample Size}

1. Draw a sample of size 10.  Note the p-value, and write a conclusion based on this p-value.  

2. Draw a sample of size 100.  Note the p-value, and write a conclusion based on this p-value. 

3.  What generalization can be made regarding sample size and test outcome after observing what happened in the previous questions? Note: all means and standard deviations are similar among the three groups.




\newpage


\section{Pre-Questions for Students in 101}

1. Explain what it means to have $90\%$ confidence in a confidence interval.


2. As we increase the confidence level of a confidence interval what happens to the width of the interval?



\section{Post-Questions for Students in 101}

1. Explain what it means to have $90\%$ confidence in a confidence interval.


2. As we increase the confidence level of a confidence interval what happens to the width of the interval?


3. Comments on the Shiny app for confidence level:


4. What is the purpose of an ANOVA test?


5. Comments on the Shiny app for ANOVA tests:




\section{Pre-Questions for Students in 400 Level Stats}

1. What is the y-intercept of a line in words?

2. What is the slope of a line in words?


\section{Post-Questions for Students in 400 Level Stats}

1.  What is the y-intercept interpretation for a line with an intercept of 2?

2. What is the interpretation of the slope for a line with a slope of -3?

3. Comments of the Shiny app section for linear regression.


4. Explain the difference between a sample distribution and a sampling distribution.


5. Comments on the Shiny app for a sampling distribution for means.






\end{document}  
